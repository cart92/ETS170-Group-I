\documentclass[12pt]{article}
\usepackage{geometry} % see geometry.pdf on how to lay out the page. There's lots.
\geometry{a4paper} % or letter or a5paper or ... etc
% \geometry{landscape} % rotated page geometry

\usepackage{graphicx}
\usepackage[english]{babel}
\usepackage{enumitem}%can be used for automatic numbering of requirements.
\usepackage{paralist}

\newlength{\testlabellength}
\settowidth{\testlabellength}{Test 100.10.10}
% Use this when you don't have a subsection in the listed requirements.
\newenvironment{reqlist}{\begin{enumerate}[label=\bfseries Req \thesection.\arabic* , labelindent=0pt, labelwidth=\testlabellength , leftmargin=3cm]}{\end{enumerate}}
% Use this when you have a subsection in the listed requirements.
\newenvironment{reqsublist}{\begin{enumerate}[label=\bfseries Req \thesubsection.\arabic* , labelindent=0pt, labelwidth=\testlabellength , leftmargin=3cm]}{\end{enumerate}}

\date{} % delete this line to display the current date

% Begin document.
\begin{document}

\setlength{\parindent}{0em}
%==================================
% F�rs�ttsblad. 
\begin{titlepage}

\begin{minipage}{0.5\textwidth}
\begin{flushleft} % Responsible persons, write on separate lines
\textit{Lund University\\ ETS170: Requirements Engineering}\\
\end{flushleft}
\end{minipage}
~
\begin{minipage}{0.4\textwidth}
\begin{flushright}
\today

\end{flushright}
\end{minipage}\\[3cm]

	\centering
	{\scshape\LARGE ShopMate \par}
	\vspace{0.5cm}
	\rule{1\textwidth}{.6pt} \par
	{\huge\bfseries Requirements Specification\par}
	\rule{1\textwidth}{.6pt} \par
	\vspace{2cm}
%	{\Large\itshape }

%==================================
% The authors down to the left. 
\vfill
\begin{flushleft}
	\textit{By Group I:}\\
	Daniel Dornl\"{o}v\par 
	David Cartbo\par
	Jonathan Lundholm\par 
	Kristoffer Hilmersson\par
	Marcus Hilliges\par 
	Thomas Strahl\par
	\end{flushleft}
\end{titlepage}
\newpage

\tableofcontents
\newpage
\begin{center}

%==================================
% The version history of this document. 
\textit{\large Version History}

    \begin{tabular}{ | l | l | l | p{5cm} |}
    \hline
    \textbf{Version}	& \textbf{Date}		& \textbf{Description}			\\ \hline
    0.1			& 2015-11-22 			& First release			\\ \hline
   	0.2			& 2015-12-7			& Release 2 			\\ \hline
    \end{tabular}
\end{center}

%==================================
% Introduction
\section*{Introduction}


%==================================
% Business Goals
\subsection*{Business Goals}

%==================================
% Business Goals

\subsection*{Terminology}
	\begin{description}
	\setlength{\itemsep}{0pt}
	\item[The app] is referring to the ordered application for a shopping mall, i.e. the ordered product.
	\item[The user] is referring to the person using the application on their device.
	\item[Device] is referring to the unit the user is running the app on, e.g. mobile phone and smart-watch.
	\item [Offers] are store-products or services that are being sold att a lower price in the stores a the mall. Offers are displayed on the user's screen of their devices.
	\item [The server/database] is where all data about stores, offers, store-products, maps and so on are stored.
	\item [Store] refers to companies located in the mall. 
	\item [Store-Product] refers to items/products being sold in the stores, not the product that is being developed.
	\item [Location] refers to the location of a user or a store in the shopping mall. 
	\item [Shopping list] refers to a list with items or store-products that the user has added to a list and intends to buy while at the mall.
\end{description}

\subsection{Context Diagram}
\begin{figure}[h]
    \centering
    \includegraphics[width=0.5\textwidth]{../images/Context_Diagram.png}
    \caption{An activity diagram showing how the client process an incoming image.}
    \label{fig:image}
\end{figure}

%==================================
% This is where the requirements start.
%==================================

%==================================
% Goal Level. 
\section*{Goal Level}

\textbf{Business --- REQ} \\ Enter description. \\

\underline{Business --- REQ} \\ Enter description. \\

\underline{ \textbf{Business --- REQ}} \\ Enter description. \\

\underline{ \textbf{Business --- REQ}} \\ Enter description. \\

\underline{ \textbf{Business --- REQ}} \\ Enter description. \\
%==================================
% Requirements on Domain Level. 
\section*{Domain Level}

\underline{ \textbf{Data--- REQ}} \\ Enter description. \\

\underline{ \textbf{Data--- REQ}} \\ Enter description. \\

\underline{ \textbf{Navigation--- REQ}} \\ Enter description. \\

\subsection{Parking}
%==================================
% Requirements on Product Level. 
\section*{Goal Level}


%==================================
% Requirements on Design Level. 
\section*{Design Level}

\end{document}